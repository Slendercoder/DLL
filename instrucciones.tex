\documentclass{article}
\usepackage[utf8]{inputenc}
\usepackage[spanish]{babel}
\usepackage{multicol}
\usepackage[letter,left=1cm,right=1cm,top=1.5cm,bottom=1cm]{geometry}

\title{Instrucciones}
\author{}
\date{}

\begin{document}

\thispagestyle{empty}

\begin{multicols}{2}{
\noindent Los participantes de este juego son dos jugadores humanos (usted y un compañero remoto). El objetivo del juego es obtener la mayor cantidad de monedas activando la máquina dispensadora con una pareja de objetos. Para formar parejas usted tendrá a su disposición varios tipos de objetos, entre los cuales se encuentran los xol, los dup y los zab. La cantidad de monedas depende de los tipos de objeto (sin importar el orden), de acuerdo a las siguientes combinaciones:
%
\begin{itemize}
\item Un Xol con un Xol da 1 moneda
\item Un Dup con un Dup da 1 moneda
\item Un Zab con un Zab da 1 moneda
\item Un Zab con un Xol da 5 monedas
\item Un Dup con un Zab da 5 monedas
\item Un Dup con un Xol da 5 monedas
\end{itemize}
%
En la parte izquierda de la pantalla está el ``toolbox", el cual despliega los objetos que usted tiene a su disposición. Allí también encuentra la caja de envíos, la cual despliega los objetos que usted reciba de su compañero.

\vspace{0.5\baselineskip}

\noindent En la parte derecha de su ventana usted encontrará el botón ``Enviar mensaje", mediante el cual podrá interactuar con su compañero. Cuando su compañero lea su mensaje, tendrá la oportunidad de enviarle a usted un objeto.

\vspace{0.5\baselineskip}

\noindent Si su compañero le envía un mensaje, recibirá una notificación mediante un pop-up. Utilice la lista desplegable de mensajes que se encuentra en la parte derecha de su ventana para responder al mensaje.

\vspace{0.5\baselineskip}

\noindent El juego consta de 15 rondas. Cada ronda tiene un umbral de puntos; si usted supera dicho umbral, obtendrá una bonificación al final del juego.
}\end{multicols}

\

\begin{multicols}{2}{
\noindent Los participantes de este juego son dos jugadores humanos (usted y un compañero remoto). El objetivo del juego es obtener la mayor cantidad de monedas activando la máquina dispensadora con una pareja de objetos. Para formar parejas usted tendrá a su disposición varios tipos de objetos, entre los cuales se encuentran los xol, los dup y los zab. La cantidad de monedas depende de los tipos de objeto (sin importar el orden), de acuerdo a las siguientes combinaciones:
%
\begin{itemize}
\item Un Xol con un Xol da 1 moneda
\item Un Dup con un Dup da 1 moneda
\item Un Zab con un Zab da 1 moneda
\item Un Zab con un Xol da 5 monedas
\item Un Dup con un Zab da 5 monedas
\item Un Dup con un Xol da 5 monedas
\end{itemize}
%
En la parte izquierda de la pantalla está el ``toolbox", el cual despliega los objetos que usted tiene a su disposición. Allí también encuentra la caja de envíos, la cual despliega los objetos que usted reciba de su compañero.

\vspace{0.5\baselineskip}

\noindent En la parte derecha de su ventana usted encontrará el botón ``Enviar mensaje", mediante el cual podrá interactuar con su compañero. Cuando su compañero lea su mensaje, tendrá la oportunidad de enviarle a usted un objeto.

\vspace{0.5\baselineskip}

\noindent Si su compañero le envía un mensaje, recibirá una notificación mediante un pop-up. Utilice la lista desplegable de mensajes que se encuentra en la parte derecha de su ventana para responder al mensaje.

\vspace{0.5\baselineskip}

\noindent El juego consta de 15 rondas. Cada ronda tiene un umbral de puntos; si usted supera dicho umbral, obtendrá una bonificación al final del juego.
}\end{multicols}

\

\begin{multicols}{2}{
\noindent Los participantes de este juego son dos jugadores humanos (usted y un compañero remoto). El objetivo del juego es obtener la mayor cantidad de monedas activando la máquina dispensadora con una pareja de objetos. Para formar parejas usted tendrá a su disposición varios tipos de objetos, entre los cuales se encuentran los xol, los dup y los zab. La cantidad de monedas depende de los tipos de objeto (sin importar el orden), de acuerdo a las siguientes combinaciones:
%
\begin{itemize}
\item Un Xol con un Xol da 1 moneda
\item Un Dup con un Dup da 1 moneda
\item Un Zab con un Zab da 1 moneda
\item Un Zab con un Xol da 5 monedas
\item Un Dup con un Zab da 5 monedas
\item Un Dup con un Xol da 5 monedas
\end{itemize}
%
En la parte izquierda de la pantalla está el ``toolbox", el cual despliega los objetos que usted tiene a su disposición. Allí también encuentra la caja de envíos, la cual despliega los objetos que usted reciba de su compañero.

\vspace{0.5\baselineskip}

\noindent En la parte derecha de su ventana usted encontrará el botón ``Enviar mensaje", mediante el cual podrá interactuar con su compañero. Cuando su compañero lea su mensaje, tendrá la oportunidad de enviarle a usted un objeto.

\vspace{0.5\baselineskip}

\noindent Si su compañero le envía un mensaje, recibirá una notificación mediante un pop-up. Utilice la lista desplegable de mensajes que se encuentra en la parte derecha de su ventana para responder al mensaje.

\vspace{0.5\baselineskip}

\noindent El juego consta de 15 rondas. Cada ronda tiene un umbral de puntos; si usted supera dicho umbral, obtendrá una bonificación al final del juego.
}\end{multicols}
\end{document}
