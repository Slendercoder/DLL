\documentclass{article}
\usepackage[utf8]{inputenc}
\usepackage[spanish]{babel}
\usepackage{multicol}
\usepackage[letterpaper,left=1cm,right=1cm,top=1.5cm,bottom=1cm]{geometry}

\title{Instrucciones}
\author{}
\date{}

\begin{document}

\thispagestyle{empty}

\begin{multicols}{2}{
\noindent Los participantes de este juego son dos jugadores humanos\linebreak (usted y un compañero remoto). El objetivo del juego es\linebreak obtener la mayor cantidad de monedas posible, para lo cual tendrá que formar parejas de objetos que activan una máquina dispensadora. Para formar parejas usted tendrá a su disposición varios tipos de objetos, entre los cuales se encuentran los xol, los dup y los zab. 

\vspace{0.5\baselineskip}

\noindent La cantidad de monedas depende de los tipos de objeto que conforman las parejas (sin importar el orden), de acuerdo a las siguientes combinaciones:

\vspace{-0.25\baselineskip}

\begin{tabular}{ccc}\\
Xol con Xol: 1 moneda && Zab con Xol: 5 monedas\\
Dup con Dup: 1 moneda && Dup con Zab: 5 monedas\\
Zab con Zab: 1 moneda && Dup con Xol: 5 monedas
\end{tabular}

\vspace{0.5\baselineskip}

\noindent En la parte izquierda de la pantalla está el ``toolbox", el cual despliega los objetos que usted tiene a su disposición. Allí también encuentra la caja de objetos recibidos, la cual despliega los objetos que usted reciba de su compañero.

\vspace{0.5\baselineskip}

\noindent En la parte derecha de su ventana usted encontrará el botón ``Enviar mensaje", mediante el cual podrá interactuar con su compañero. Cuando su compañero lea su mensaje, tendrá la oportunidad de enviarle a usted un objeto. 

\vspace{0.5\baselineskip}

\noindent Si su compañero le envía un mensaje, recibirá una notificación mediante un pop-up. Utilice la lista desplegable de mensajes que se encuentra en la parte derecha de su ventana para responder al mensaje.

\vspace{0.5\baselineskip}

\noindent El juego consta de 15 rondas. Cada ronda tiene un umbral de puntos; si usted supera dicho umbral, obtendrá una bonificación al final del juego.
}\end{multicols}

\ \\

\begin{multicols}{2}{
\noindent Los participantes de este juego son dos jugadores humanos\linebreak (usted y un compañero remoto). El objetivo del juego es\linebreak obtener la mayor cantidad de monedas posible, para lo cual tendrá que formar parejas de objetos que activan una máquina dispensadora. Para formar parejas usted tendrá a su disposición varios tipos de objetos, entre los cuales se encuentran los xol, los dup y los zab. 

\vspace{0.5\baselineskip}

\noindent La cantidad de monedas depende de los tipos de objeto que conforman las parejas (sin importar el orden), de acuerdo a las siguientes combinaciones:

\vspace{-0.25\baselineskip}

\begin{tabular}{ccc}\\
Xol con Xol: 1 moneda && Zab con Xol: 5 monedas\\
Dup con Dup: 1 moneda && Dup con Zab: 5 monedas\\
Zab con Zab: 1 moneda && Dup con Xol: 5 monedas
\end{tabular}

\vspace{0.5\baselineskip}

\noindent En la parte izquierda de la pantalla está el ``toolbox", el cual despliega los objetos que usted tiene a su disposición. Allí también encuentra la caja de objetos recibidos, la cual despliega los objetos que usted reciba de su compañero.

\vspace{0.5\baselineskip}

\noindent En la parte derecha de su ventana usted encontrará el botón ``Enviar mensaje", mediante el cual podrá interactuar con su compañero. Cuando su compañero lea su mensaje, tendrá la oportunidad de enviarle a usted un objeto. 

\vspace{0.5\baselineskip}

\noindent Si su compañero le envía un mensaje, recibirá una notificación mediante un pop-up. Utilice la lista desplegable de mensajes que se encuentra en la parte derecha de su ventana para responder al mensaje.

\vspace{0.5\baselineskip}

\noindent El juego consta de 15 rondas. Cada ronda tiene un umbral de puntos; si usted supera dicho umbral, obtendrá una bonificación al final del juego.
}\end{multicols}

\ \\

\begin{multicols}{2}{
\noindent Los participantes de este juego son dos jugadores humanos\linebreak (usted y un compañero remoto). El objetivo del juego es\linebreak obtener la mayor cantidad de monedas posible, para lo cual tendrá que formar parejas de objetos que activan una máquina dispensadora. Para formar parejas usted tendrá a su disposición varios tipos de objetos, entre los cuales se encuentran los xol, los dup y los zab. 

\vspace{0.5\baselineskip}

\noindent La cantidad de monedas depende de los tipos de objeto que conforman las parejas (sin importar el orden), de acuerdo a las siguientes combinaciones:

\vspace{-0.25\baselineskip}

\begin{tabular}{ccc}\\
Xol con Xol: 1 moneda && Zab con Xol: 5 monedas\\
Dup con Dup: 1 moneda && Dup con Zab: 5 monedas\\
Zab con Zab: 1 moneda && Dup con Xol: 5 monedas
\end{tabular}

\vspace{0.5\baselineskip}

\noindent En la parte izquierda de la pantalla está el ``toolbox", el cual despliega los objetos que usted tiene a su disposición. Allí también encuentra la caja de objetos recibidos, la cual despliega los objetos que usted reciba de su compañero.

\vspace{0.5\baselineskip}

\noindent En la parte derecha de su ventana usted encontrará el botón ``Enviar mensaje", mediante el cual podrá interactuar con su compañero. Cuando su compañero lea su mensaje, tendrá la oportunidad de enviarle a usted un objeto. 

\vspace{0.5\baselineskip}

\noindent Si su compañero le envía un mensaje, recibirá una notificación mediante un pop-up. Utilice la lista desplegable de mensajes que se encuentra en la parte derecha de su ventana para responder al mensaje.

\vspace{0.5\baselineskip}

\noindent El juego consta de 15 rondas. Cada ronda tiene un umbral de puntos; si usted supera dicho umbral, obtendrá una bonificación al final del juego.
}\end{multicols}
\end{document}
